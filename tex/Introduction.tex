\chapter{Introduction}

\begin{itemize}
\item
cold atoms toolbox, controllability 

\item
optical lattices, quantum simulations, Mott as platform for many experiments

\item
many setups/experiments require low temperature/few defects, slow experiments vulnerable to decoherence and heating -> state preparation and dynamical control is important and made hard by phase transition

\item
optimal control is framework for manipulation of systems and expressing above problem as a minimization problem

\item 
simulating many-body systems is exponentially expensive -> tensor networks 

\item
this thesis presents a general framework for optimizing state transfers in lattice systems. The framework combines GROUP control, tensor networks, and interior point methods. Framework tested on non-integrable Bose-Hubbard model
\end{itemize}

Over the last two decades great improvements in the experimental cold atoms toolbox have enabled the study of countless quantum systems and phenomena. Ultracold quantum gases are extremely versatile, as a wide range of Hamiltonians can be mapped unto the system allowing various theoretical models to be realized experimentally. Furthermore, a very high degree of control over the systems is achievable through the manipulation of external fields.
A particularly interesting technique is loading the cold atoms into optical lattices created by the interference of laser beams. In a lattice, the cold gases can be used for quantum simulations of solid-state systems, quantum logic gate operations, single-atom transistors, and much more. Many of these applications require the system initially being in a Mott-insulating state, where the atoms are evenly distributed and well-localized within the sites of the lattice.
Due to the fragile nature of quantum effects, extremely low temperature are necessary in order to experimentally observe these phenomena. Furthermore, many applications of lattice systems require the preparation of certain configurations, such as the Mott, with a minimum of defects. Although the manipulation of cold gases is easily achieved, exercising full control over the systems is very difficult due to the complexity of the many-body dynamics. The preparation of high-quality Mott-insulators is particularly challenging, as the experimental procedure requires the crossing of a quantum phase transition, at which various mechanisms introduces disorder to the system. Currently, many experiments rely on sub-optimal control protocols, which often causes unnecessary heating of the system while being needlessly slow thus making the system vulnerable to decoherence.
Therefore, the preparation of states and the dynamical control of systems are some of the central challenges in current cold atoms physics.

Quantum optimal control theory is a framework for designing control schemes for quantum systems. 

