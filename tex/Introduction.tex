\chapter{Introduction}
Over the last two decades great improvements in the experimental cold atoms toolbox have enabled the study of countless quantum systems and phenomena \cite{manybodyBloch,Bloch2012}. Ultracold quantum gases are extremely versatile, as a wide range of Hamiltonians can be mapped unto the system allowing various theoretical models to be realized experimentally. Furthermore, a very high degree of control over the systems is achievable through the manipulation of external fields \cite{JakschZoller}.
As the experimental techniques have been improved, focus has shifted towards controlling quantum systems rather than simply measuring them. Exercising precise control over quantum systems is the key for quantum-based technologies of the future.

A powerful type of system for manipulating quantum effects can be realized by loading cold atoms into optical lattices created by the interference of laser beams \cite{grimm}. In a lattice, the cold gases can be used for quantum simulations \cite{Jane2003,Jaksch2003}, quantum logic gate operations \cite{Zoller1999,Mandel2003,Jaksch2000}, single-atom transistors \cite{Micheli2004}, and many other promising applications for future technology. Many of these operations require the system initially being in a Mott-insulating state, where the atoms are evenly distributed and well-localized within the lattice \cite{lewenstein}.
Due to the fragile nature of quantum effects, extremely low temperature are necessary in order to experimentally observe these phenomena. Furthermore, many applications of lattice systems require the preparation of certain configurations, such as the Mott, with a minimum of defects. Exercising full control over lattice systems is very difficult due to the complexity of the many-body dynamics. The preparation of high-quality Mott-insulators is particularly challenging, as the experimental procedure requires the crossing of a quantum phase transition, where various mechanisms introduce disorder to the system \cite{Zurek2005,Braun2015}. Currently, many experiments rely on sub-optimal control protocols, which often end up causing unnecessary heating of the system while being needlessly slow, thus making the system vulnerable to decoherence.
Therefore, the preparation of states and the dynamical control of systems are some of the central challenges in current cold atoms physics.

Quantum optimal control theory is a framework for designing control schemes for quantum systems \cite{Peirce1988,Werschnik2007}. The desired dynamics are achieved by formulating the control problem as the minimization of some cost function, whereby well-established methods from mathematical optimization theory can be employed. Additionally, through the optimization landscape much information can be inferred about the controllability of the system \cite{Rabitz2004}. Optimal control problems are often subject to a series of constraints due to physical and technical limitations. Furthermore, the dynamics of complex systems are often highly non-linear. Therefore, the optimal control functions responsible for generating the desired dynamics can be found analytically for only very few quantum systems, whereby one often has to rely on numerical simulations of the system. Hence, algorithms for optimizing the control functions are central in quantum optimal control theory. These algorithms can roughly be divided into those utilizing derivatives of the cost function \cite{Khaneja2005,Krotov1995} versus those relying on classical Nelder-Mead methods \cite{Doria2011}.

The effectiveness of quantum optimal control is typically limited by the difficulty of simulating quantum systems \cite{Vidal2003}. Quantum many-body systems, such as cold gases in optical lattices, have Hilbert spaces which scale exponentially with the system size. Therefore, simulating even small systems is extremely resource demanding. In one-dimensional lattice systems, quantum states and operators can be decomposed as tensor networks. These techniques are incredibly powerful, as the networks by construction puts an upper bound on the entanglement between different domains of the system, whereby most states of the Hilbert space can be neglected \cite{schollwock,Cramer}. Thus, tensor networks provide a low-entanglement effective theory of the system, which remains valid even at longer durations due to a finite spreading velocity of correlations \cite{Bravyi2006,Eisert2006}.\\

This thesis presents a framework for finding optimal control schemes for precise, dynamical manipulation of one-dimensional lattice systems. The framework combines the tensor network description of quantum states and operators with derivative-based optimal control algorithms. The tensorial computation of cost-gradients has previously been argued to be too resource demanding and thus impossible \cite{Doria2011}, however, the results presented here prove that this is not the case. In fact, when compared to literature methods, the framework presented here performs much better. Its high performance can partly be attributed to the use of interior point methods, which are gradient-based optimization techniques for non-linear, constrained problems. Interior point methods are especially applicable to control problems, as they converge using only a minimal number of the otherwise computationally expensive gradient evaluations \cite{wright}.

The framework can in principle be applied to any quantum many-body system in an optical lattice. Here, the Bose-Hubbard model is examined with the goal of preparing a high-quality Mott-insulating state. This requires transferring the cold atoms from the initial superfluid phase, whereby the system has to undergo a phase transition. Controlling a system during such a phase transition is very difficult due to complex quantum phenomena emerging near the critical point. Hence, non-optimized experimental control sequences produce Mott-insulators quite inefficiently. Therefore, numerical optimizations are a necessary step towards realizing quantum technologies.\\
Thus, the thesis is outlined as follows:
\begin{itemize}
\item[]
\textit{Chapter 2} covers the basic theory of ultracold, neutral atoms in optical lattices. The chapter introduces the Bose-Hubbard model along with its phases, and a mean-field phase diagram of the model is derived.

\item[]
\textit{Chapter 3} details the parameterization of quantum states via tensor networks. These networks enables a resource-efficient description of quantum many-body systems, like the Bose-Hubbard model. Without tensor networks, numerical descriptions of the Bose-Hubbard model are highly limited due to its exponentially large Hilbert space.

\item[]
\textit{Chapter 4} introduces tensor network algorithms for ground state searches and time evolution. Here, an efficient time-evolution algorithm tailored to the Bose-Hubbard model is presented.

\item[]
\textit{Chapter 5} discusses quantum optimal control theory. Here, algorithms for conducting gradient-based optimal control are derived, and the fundamental theory of interior point methods is presented.

\item[]
\textit{Chapter 6} presents a framework for optimal control of lattice systems, which combines elements from all previous chapters. The framework is applied to the Bose-Hubbard model, and its performance is compared with literature methods. Next, non-equilibrium dynamics in large Bose-Hubbard systems are analyzed, and the application of the framework on such systems is discussed.     
\end{itemize}
