\chapter{Introduction}
Over the last two decades great improvements in the experimental cold atoms toolbox have enabled the study of countless quantum systems and phenomena. Ultracold quantum gases are extremely versatile, as a wide range of Hamiltonians can be mapped unto the system allowing various theoretical models to be realized experimentally. Furthermore, a very high degree of control over the systems is achievable through the manipulation of external fields.
A particularly interesting technique is loading the cold atoms into optical lattices created by the interference of laser beams. In a lattice, the cold gases can be used for quantum simulations of solid-state systems, quantum logic gate operations, single-atom transistors, and much more. Many of these applications require the system initially being in a Mott-insulating state, where the atoms are evenly distributed and well-localized within the sites of the lattice.
Due to the fragile nature of quantum effects, extremely low temperature are necessary in order to experimentally observe these phenomena. Furthermore, many applications of lattice systems require the preparation of certain configurations, such as the Mott, with a minimum of defects. Although the manipulation of cold gases is easily achieved, exercising full control over the systems is very difficult due to the complexity of the many-body dynamics. The preparation of high-quality Mott-insulators is particularly challenging, as the experimental procedure requires the crossing of a quantum phase transition, at which various mechanisms introduces disorder to the system. Currently, many experiments rely on sub-optimal control protocols, which often causes unnecessary heating of the system while being needlessly slow thus making the system vulnerable to decoherence.
Therefore, the preparation of states and the dynamical control of systems are some of the central challenges in current cold atoms physics.

Quantum optimal control theory is a framework for designing control schemes for quantum systems. The desired dynamics are achieved by formulating the control problem as the minimization of some cost function, whereby well-established methods from mathematical optimization theory can be employed. Additionally, through the optimization landscape much information can be inferred about the controllability of the system. Optimal control problem are often subject to a series of constraints due to physical and technical limitations. Furthermore, the dynamics of complex systems are often highly non-linear. Therefore, the optimal control functions responsible for generating the desired dynamics can be found analytically for only very few quantum systems. Thus, one often has to rely on numerical simulations of the system, whereby algorithms for optimizing the control functions are central in quantum optimal control theory. These algorithms can roughly be divided into those utilizing derivatives of the cost function versus those relying on classical Nelder-Mead methods.

The effectiveness of quantum optimal control is often limited by the difficulty of simulating quantum systems. Quantum many-body systems, such as cold gases in optical lattices, often have Hilbert spaces which scale exponentially with the system size. Therefore, simulating even small systems is extremely resource demanding. In one-dimensional lattice systems, quantum states and operators can be decomposed as tensor networks. These techniques are incredibly powerful, as the networks by construction puts an upper bound on the entanglement between different domains of the system, whereby most states of the Hilbert space can be neglected. Thus, tensor networks provide a low-entanglement effective theory of the system, which remains valid at even longer durations due to the finite spreading velocity of correlations within the lattice systems.\\

This thesis presents a framework for finding optimal control schemes for precise, dynamical manipulation of one-dimensional lattice systems. The framework combines the tensor network description of quantum states and operators with derivative-based optimal control algorithms. While the tensorial computation of the gradient of the cost function has previously been argued to be too resource demanding and thus impossible, the results presented here prove that this is not the case. In fact, when compared to tensor-network-based optimizations utilizing the Nelder-Mead method, the framework presented here performs much better. The high performance can partly be attributed to the use of interior point methods, which are gradient-based optimization techniques for non-linear, constrained problems. Interior point methods are especially applicable to control problems, as they converge using only a minimal number of the otherwise computationally expensive gradient evaluations

To demonstrate the framework, a ground state transfer from the superfluid to the Mott-insulating phase of the Bose-Hubbard model is optimized. Simulating Bose-Hubbard systems is considered fairly difficult due to the non-integrability of the model. Furthermore, the state transfer examined is very relevant for experimental purposes, whereby it is an ideal benchmark for the framework. Although originally developed with the Bose-Hubbard model in mind, the framework can easily be applied to any one-dimensional quantum lattice system. 

