\chapter{Diagrammatic Representation of Matrix Product States} \label{chap:diagrams}

Due to the many indices, equations describing contractions of matrix product states are often quite hard to read. Therefore, the equations are often represented graphically through diagrams. Many variations of diagrams exists, however, they all follow some general rules:
\begin{itemize}
\item
Tensors are represented by nodes.
\item
Indices are represented by lines. Horisontal lines are \textit{bond indices}, while vertical lines are \textit{physical indices}.
\item
Tensors connected by a line are contracted over said bond.
\end{itemize}
Through these basic rules, most equations involving matrix product states can be expressed through diagrams. In this instance, the color or shape of a node provides information regarding the properties of the tensor. Therefore, the tensors shown in this thesis follow these guidelines:
\begin{itemize}
\item
Tensors part of an MPS are square-like. The color of the tensor denotes its normalisation, as shown in figure \ref{fig:MPStensors}. These different types of normalisation are part of of MPS canonical forms (see Section \ref{sec:canonical}).
\item
Tensors part of MPO's are grey and circular/elliptical depending on the number of sites they span over. These tensors have two vertical legs compared two the one leg of the MPS tensors.
\item
Matrices (tensors without any physical index) are depicted as a diamond. These often occur after a singular value decomposition. 
\end{itemize} 


\begin{figure}[h!]
\centering % <-- add this
\begin{subfigure}[t]{0.2\textwidth}
	\caption{}  	
  	\begin{tikzpicture}[inner sep=1mm]
	\node[tensor, label={\small $M^{[n]}$}] (tensor1) at (0, 0) {};
	\node (index1) at (0, -1) {$j_n$};
	\node (index2) at (1, 0) {$\alpha_{n}$};
	\node (index3) at (-1.2, 0) {$\alpha_{n-1}$};
	
	\draw[-] (tensor1) -- (index1);
	\draw[-] (tensor1) -- (index2);
	\draw[-] (tensor1) -- (index3);
\end{tikzpicture}
\end{subfigure}
\hspace{5mm}
\begin{subfigure}[t]{0.2\textwidth}    
	\caption{}  	
  	\begin{tikzpicture}[inner sep=1mm]
	\node[tensorl, label={\small $A^{[n]}$}] (tensor1) at (0, 0) {};
	\node (index1) at (0, -1) {};
	\node (index2) at (1, 0) {};
	\node (index3) at (-1, 0) {};
	
	\draw[-] (tensor1) -- (index1);
	\draw[-] (tensor1) -- (index2);
	\draw[-] (tensor1) -- (index3);
\end{tikzpicture}
\end{subfigure}
\hspace{5mm}
\begin{subfigure}[t]{0.2\textwidth}    
	\caption{}  	
  	\begin{tikzpicture}[inner sep=1mm]
	\node[tensorr, label={\small $B^{[n]}$}] (tensor1) at (0, 0) {};
	\node (index1) at (0, -1) {};
	\node (index2) at (1, 0) {};
	\node (index3) at (-1, 0) {};
	
	\draw[-] (tensor1) -- (index1);
	\draw[-] (tensor1) -- (index2);
	\draw[-] (tensor1) -- (index3);
\end{tikzpicture}
\end{subfigure}
\hspace{5mm}
\begin{subfigure}[t]{0.2\textwidth}    
	\caption{}  	
  	\begin{tikzpicture}[inner sep=1mm]
	\node[tensorc, label={\small $\Psi^{[n]}$}] (tensor1) at (0, 0) {};
	\node (index1) at (0, -1) {};
	\node (index2) at (1, 0) {};
	\node (index3) at (-1, 0) {};
	
	\draw[-] (tensor1) -- (index1);
	\draw[-] (tensor1) -- (index2);
	\draw[-] (tensor1) -- (index3);
\end{tikzpicture}
\end{subfigure}
\caption{\textit{The four different types of MPS tensors used in diagrams. \textbf{(a)} General tensor of un-specified normalisation. The indices corresponding to the tensor are labeled. The remaining tensors are: Left-normalised \textbf{(b)}, right-normalised \textbf{(c)}, central cite \textbf{(d)}. }}
\label{fig:MPStensors}
\end{figure}
	
\section*{Contraction of Tensor Networks}
Similarly to matrix multiplication, the number of computations need for contracting a string of tensors can be greatly reduced when performing the contractions in the right order. Consider the general case of two general states $\ket{\psi}$ and $\ket{\phi}$ described by the matrices $M$ and $\tilde{M}$. The overlap between these states reads
\begin{equation}
	\braket{\phi | \psi} = \sum_{j_1 , \ldots , j_L} \tilde{M}^{j_L \dag} \ldots \tilde{M}^{j_1 \dag} M^{j_1} \ldots M^{j_L} \; , 
	\label{eq:overlap}
\end{equation}
which is represented diagrammatically in figure \ref{fig:effCont}.
\begin{figure}[h!]
	\centering
	\begin{tikzpicture}[inner sep=1mm]
	\foreach \i in  {1,...,5} {
		\node[tensor] (t\i) at (\i*2-1, 0) {};
		\node[tensor] (b\i) at (\i*2-1, -2) {};
		\draw[-] (t\i) -- (b\i);	
	};
	\foreach \i in  {1,...,4} {
		\pgfmathtruncatemacro{\iplusone}{\i + 1};
		\draw[-] (t\i) -- (t\iplusone);
		\draw[-] (b\i) -- (b\iplusone);
	};
	
	\node (node1) at (0.7, -1) {\textbf{1}};
	\node (node2) at (2, 0.3) {\textbf{2}};
	\node (node3) at (2, -2.3) {\textbf{3}};
	\node (node4) at (2.7, -1) {\textbf{4}};
	
	\node (psi) at (10, 0) {$\ket{\psi}$};
	\node (phi) at (10, -2) {$\bra{\phi}$};
\end{tikzpicture}
	\caption{\textit{Efficient contraction of the overlap between two general states $\ket{\psi}$ and $\ket{\phi}$. By contracting the bonds in the specified order the number of multiplications is greatly reduced.}}
	\label{fig:effCont}
\end{figure}
The evaluation of the overlap can be drastically sped up by considering the optimal order of contractions, which corresponds to an optimal bracketing of eq. \eqref{eq:overlap}:
\begin{equation}
	\braket{\phi | \psi} = \sum_{j_L} \tilde{M}^{j_L \dag} \left( \ldots \left( \sum_{j_2} \tilde{M}^{j_2 \dag} \left( \sum_{j_1} \tilde{M}^{j_1 \dag} M^{j_1} \right) M^{j_2} \right) \ldots \right) M^{j_L} \; .
	\label{eq:optBrackets}
\end{equation}  
Within the innermost bracket a matrix is formed by multiplying a column and a row vector followed by the summation over the first physical index, $j_1$. In the remaining set of brackets three matrices are multiplied and the corresponding physical index is summed over. However, after the first contraction the complexity of the operation does not increase. Consider the worst case scenario of all the matrices being of dimension $(D \times D)$ with a local Hilbert space of dimension $d$. The total operational cost of the optimal contraction is $\mathcal{O}(L D^3 d)$ compared to otherwise exponential complexity of a random order of contractions \cite{schollwock}. The order of contractions described in eq. \eqref{eq:optBrackets} is illustrated in figure \ref{fig:effCont}.\\
When calculating a norm, $\braket{\psi | \psi}$, the canonical form of the MPS greatly reduces to computational time, as having an either left- or right-canonical MPS implies a norm of 1 without any contractions needed.